\documentclass{article}
\usepackage[T1]{fontenc}
\usepackage[a4paper, landscape]{geometry}
\usepackage{hyperref}
\usepackage{nopageno}
\usepackage{parskip}

\begin{document}
% The hyperlink doesn't work but that's fine
\textbf{The last}
\href{https://www.unicornsparty.com/}{\textbf{Unicorns} Party} was when
I reconnected with a person who has been very special to me in these
couple of years.  They gave me pointers at a time when I was feeling
lost. I wouldn't have met so many people and friends, been to places
previously unheard of, nor would I have understood myself the way I do
now, if it was not for them. For that, I am grateful. I thought I shall
write a piece dedicated to the moment where we saw each other again.

\section*{Performance Instruction}
The piece can be performed as it is, for solo piano, fully notated.
%
It can also be performed as a mix of playing the notated music and
improvising.
%
In this case, the notated music can serve as a guide for the
improvisation.
%
Metric modulation indicates the transition from one section to the next.
Although not indicated, when improvising, each section can be repeated,
as many times as desired.
%
When improvising, this piece can be played accompanied.

\noindent
Accidentals apply only to the note they are attached to.

\noindent
Dynamics are notated above the staff rather than next to the notes.
\end{document}
